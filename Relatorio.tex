\documentclass[12pt]{article}

%% Escrevendo em português:
\usepackage[brazil]{babel}
\usepackage[utf8]{inputenc}
\usepackage[pdftex]{hyperref}
\usepackage{epstopdf}
\usepackage{psfrag}
\usepackage{setspace}
\usepackage{graphicx}
\usepackage{color}
\usepackage{caption}
\usepackage{subcaption}
\usepackage{float}
\usepackage{amsmath}
\usepackage{amsfonts}
\usepackage{amssymb}
\usepackage{grafcet}
%----------------------------


\newtheorem{teorema}{Teorema}%[chapter]
\newtheorem{problema}[teorema]{Problema}
\newtheorem{conj}[teorema]{Conjectura}
\newtheorem{lema}[teorema]{Lema}
\newtheorem{exerc}[teorema]{Exercício}

\definecolor{lightgray}{gray}{0.95}

% Ad hoc macros
\newcommand{\qed}       {\hfill\Box}
\newcommand{\card}[1]   {\left|#1\right|}
\newcommand{\nct}       {\chi_{_T}}
\newcommand{\cnk}[2]    {C_{#1}^{#2}}
\newcommand{\ed}[2]     {#1#2}
\newcommand{\teto}[1]   {\lceil #1 \rceil}
\newcommand{\piso}[1]   {\lfloor #1 \rfloor}
\newcommand{\itab}[1]{\hspace{0em}\rlap{#1}}
\newcommand{\tab}[1]{\hspace{.2\textwidth}\rlap{#1}}

%%%%%%%%%%%%%%%%
\begin{document}
	%%%%%%%%%%%%%%%%
	
\begin{titlepage}
\begin{center}

\newcommand{\HRule}{\rule{\linewidth}{0.5mm}}
% Upper part of the page. The '~' is needed because \\
% only works if a paragraph has started.
\includegraphics[width=0.15\textwidth]{logoUnicamp}~\\[1cm]

\textsc{\LARGE Universidade Estadual de Campinas}\\[1.5cm]

\textsc{\Large Faculdade de Engenharia Mecânica}\\[0.5cm]

% Title
\HRule \\[0.4cm]
{ \Large \bfseries{ES726 - Laboratório de Sistemas Pneumáticos e Hidráulicos\\ \vspace{0.8cm} Projeto Final}\\
\large{Partiu bar - Parte IV - A Terra e o Tempo} \\[0.4cm] }

\HRule \\[1.5cm]

% Author and supervisor
\begin{minipage}{0.6\textwidth}
\begin{flushleft} \large
\emph{Nome:}\\
Daniel Dello Russo Oliveira\\ Marcelli Tiemi Kian\\ Vinicius Ragazi David
\end{flushleft}
\end{minipage}
\begin{minipage}{0.2\textwidth}
\begin{flushright} \large
\emph{RA}\\ 101918\\
117892\\ 120258
\end{flushright}
\end{minipage}

\vfill

% Bottom of the page
{\large \today}

\end{center}
\end{titlepage}

	
	\tableofcontents
	%\listoffigures
	%\listoftables
	
	\clearpage
	
	%%%%%%%%%%%%%%%%%%%%%%%%%%%%%%%%%%%%%%%%%%
	\section{Descrição Técnica do Processo}
	
	\begin{par}
		Este relatório consiste na descrição da solução encontrada para o problema da maturação e filtragem da produção de cerveja. O processo começa após a fermentação da cerveja (cerveja verde) que são mandados para os tanques de maturação. No tanque a cerveja verde permanece entre 1h e 3h com controle constante de sua temperatura, esta necessitando estar em 0ºC, ou no máximo entre -5 e 5ºC. Este controle de temperatura deve ser feito com base num fluido refrigerante.
	\end{par}
	
	\begin{figure}[H]
		\centering
		\includegraphics [width=4in]{tanque.png}
		\caption {Tanque de maturação da cerveja verde.}
		\label{fig:corrente}
	\end{figure}
	
	\begin{par}
		Passado este tempo e com sucesso do controle de temperatura a cerveja verde torna-se cerveja madura. A próxima etapa é passar por um filtro com terra diatomácea, que retira partículas desagradáveis à cerveja. O resíduo do filtro deve ser descatado após o uso.
	\end{par}
	
	\begin{figure}[H]
		\centering
		\includegraphics [width=4in]{filtro.png}
		\caption {Filtro da cerveja maturada.}
		\label{fig:corrente}
	\end{figure}
	
	\begin{par}
		Após a filtragem a cerveja é então destinada à próxima etapa da sua fabricação, sendo este não descrito por este trabalho.
	\end{par}
	
	
	\section {Análise do Projeto}
	\begin {itemize}
	\item Modo Automático
	\begin{par}
	
	O modo automático consiste na mudança de estado automática. Quando todas as condições necessárias para a mudança de estado se tornam verdadeiras e o modo automático está ativo a mudança de estado acontecerá, sendo assim, não sendo necessária a atuação humana. Este modo permite um processo mais rápido e mais barato por não necessitar de um funcionário presente para fazer as transições. Contudo poderá haver problemas caso a verificação para as condições estiver com problema, se os sensores, por exemplo, estiverem com problema o processo pode avançar mesmo não sendo o momento apropriado para tal.
	\end{par}
	\item Modo Passo a Passo
	
	O modo passo a passo é o oposto do modo automático, sendo assim necessário a atuação humana para a transição de estados. Com todas as condições de transição verdadeiras o processo apenas mudará de estado caso um botão no IHM (interface homem máquina) seja apertado manualmente. Caso 
	
	O valor do modo passo a passo é verificado em teste, já que o processo pode ser totaltmente controlado pelo engenheiro de qualidade, testando todas as transições e funcionalidade das entradas (sensores e timers) do sistema.
	\item Modo Homming
	
	\item Parada de emergência
	
	\item Alarmes e tratamentos de Erros
	
	\item IHM
\end{itemize}

\section {Tabela de designação}

\section {Implementação do sistema}
\subsection{Grafcet}
\begin{tikzpicture}
\EtapeInit[0,0]{0}
\Transition[VX0]{0}
\DivET{T0}{-5/L10,5/L20}
\Etape[VL10]{10}
\Transition[VX10]{10}
\EtapeEncapsulante[VT10]{20}
\Transition[VX20]{20}
\SequenceEE[VL20]{11}{12}
\ConvET[5]{T20}{X12}{br3}
\Transition[br3]{3}
\Etape[VT3]{4}
\Transition[VX4]{4}
\LienRetour[10]{T4}{X0}

\begin{Encap}[nom]{7,-1.5}{20}{G2}
\Etape[0,0]{200}
\Transition[VX200]{200}
\Etape[VT200]{201}
\Transition[VX201]{201}
\LienRetour{T201}{X200}
\LienActivation{X200}
\end{Encap}

\end{tikzpicture}

\section {Conclusões}

\begin{thebibliography}{5}
	
	\bibitem{Ogata11} K. Ogata, \emph{Engenharia de Controle Moderno}, 6ª edição, 2011.
	
\end{thebibliography}

\end{document}
