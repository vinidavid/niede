\documentclass[12pt]{article}

%% Escrevendo em português:
\usepackage[brazil]{babel}
\usepackage[utf8]{inputenc}
\usepackage[pdftex]{hyperref}
\usepackage{epstopdf}
\usepackage{etoolbox}
\usepackage{psfrag}
\usepackage{setspace}
\usepackage{graphicx}
\usepackage{epsfig}
\usepackage{epstopdf}
\usepackage{color}
\usepackage{caption}
\usepackage{subcaption}
\usepackage{float}
\usepackage{amsmath}
\usepackage{amsfonts}
\usepackage{amssymb}
\usepackage{tikz}
%----------------------------


\newtheorem{teorema}{Teorema}%[chapter]
\newtheorem{problema}[teorema]{Problema}
\newtheorem{conj}[teorema]{Conjectura}
\newtheorem{lema}[teorema]{Lema}
\newtheorem{exerc}[teorema]{Exercício}

\definecolor{lightgray}{gray}{0.95}

% Ad hoc macros
\newcommand{\qed}       {\hfill\Box}
\newcommand{\card}[1]   {\left|#1\right|}
\newcommand{\nct}       {\chi_{_T}}
\newcommand{\cnk}[2]    {C_{#1}^{#2}}
\newcommand{\ed}[2]     {#1#2}
\newcommand{\teto}[1]   {\lceil #1 \rceil}
\newcommand{\piso}[1]   {\lfloor #1 \rfloor}
\newcommand{\itab}[1]{\hspace{0em}\rlap{#1}}
\newcommand{\tab}[1]{\hspace{.2\textwidth}\rlap{#1}}

%%%%%%%%%%%%%%%%
\begin{document}
%%%%%%%%%%%%%%%%

\begin{titlepage}
\begin{center}

\newcommand{\HRule}{\rule{\linewidth}{0.5mm}}
% Upper part of the page. The '~' is needed because \\
% only works if a paragraph has started.
\includegraphics[width=0.15\textwidth]{logoUnicamp}~\\[1cm]

\textsc{\LARGE Universidade Estadual de Campinas}\\[1.5cm]

\textsc{\Large Faculdade de Engenharia Mecânica}\\[0.5cm]

% Title
\HRule \\[0.4cm]
{ \Large \bfseries{ES726 - Laboratório de Sistemas Pneumáticos e Hidráulicos\\ \vspace{0.8cm} Projeto Final}\\
\large{Partiu bar - Parte IV - A Terra e o Tempo} \\[0.4cm] }

\HRule \\[1.5cm]

% Author and supervisor
\begin{minipage}{0.6\textwidth}
\begin{flushleft} \large
\emph{Nome:}\\
Daniel Dello Russo Oliveira\\ Marcelli Tiemi Kian\\ Vinicius Ragazi David
\end{flushleft}
\end{minipage}
\begin{minipage}{0.2\textwidth}
\begin{flushright} \large
\emph{RA}\\ 101918\\
117892\\ 120258
\end{flushright}
\end{minipage}

\vfill

% Bottom of the page
{\large \today}

\end{center}
\end{titlepage}


\tableofcontents
%\listoffigures
%\listoftables
 
\clearpage

\onehalfspacing
%%%%%%%%%%%%%%%%%%%%%%%%%%%%%%%%%%%%%%%%%%
\section{Descrição Técnica do Processo}

\begin{par}
Este relatório consiste na descrição da solução encontrada para o problema da maturação e filtragem da produção de cerveja. O processo começa após a fermentação da cerveja (cerveja verde) que são mandados para os tanques de maturação. No tanque a cerveja verde permanece entre 1h e 3h com controle constante de sua temperatura, esta necessitando estar em 0ºC, ou no máximo entre -5 e 5ºC. Este controle de temperatura deve ser feito com base num fluido refrigerante.

Passado este tempo e com sucesso do controle de temperatura a cerveja verde torna-se cerveja madura. A próxima etapa é passar por um filtro com terra diatomácea, que retira partículas desagradáveis à cerveja. O resíduo do filtro deve ser descatado após o uso.

Após a filtragem a cerveja é então destinada à próxima etapa da sua fabricação, sendo este não descrito por este trabalho.
\end{par}

\section {Análise do Projeto}
\begin {itemize}
\item Modo Automático

\item Modo Homming

\item Modo Passo a Passo

\item Parada de emergência

\item Alarmes e tratamentos de Erros

\item IHM
\end{itemize}

\section {Tabela de designação}

\section {Implementação do sistema}

\section {Conclusões}

\begin{thebibliography}{5}

\bibitem{Ogata11} K. Ogata, \emph{Engenharia de Controle Moderno}, 6ª edição, 2011.

\end{thebibliography}

\end{document}
    
